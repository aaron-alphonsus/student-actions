% FIRST OF ALL:
% If you are using X-based Emacs to read this file, please switch on
% Syntax Highlighting by typing:
%    ALT-X  font-lock-mode   (or META-X on X-Terminals)
%
% That should make these comments nice and red so they can be easily
% distinguished from the actual code. 
% =======================================================================

% This is a template for  Masters' Theses at WPI.
% It complies (more or less) to the standards given by the Library 
% (as of February 1999)
%
% Feel free to use this file, but I give no guarantee for its compliance
% to standards (meaning I won't pay for the paper if the library rejects it :))
%
%
% The lengths (textheight, width etc.) are fine-tuned for ps1, ps2, and ps3, 
% but seem to be somewhat dependent on the machine you are using to compile, 
% the date, time, moon phase, the weather, and other quantum effects.
% You may have to change \oddsidemargin a little, but it's about 98% correct.
%
% Also, the spacing is correct (doublespacing with footnotes correctly
% singlespaced). Curiously, the font size is not specified in the
% regulations. So feel free to change it, but the majority of theses
% that I have seen is written in 12 point font.
% 
% As for the inclusion of graphics, I recommend the methods specified
% in ``latexguide.ps'' off the CS-GSO Website. You can use other
% methods including copy and paste with a photocopier, but I think
% using the graphicx package is the easiest.
%
% Have fun and good luck with the thesis.
%
%
%  Andreas Koeller (koeller@wpi.edu)
%
%
%

% The preamble
%
%
% 12 point font, and your thesis is a ``report'' to LaTeX
\documentclass[12pt]{article}

% this enables correct linespacing and graphics inclusion via 
%``\includegraphics''
\usepackage{setspace}
\usepackage{caption}
\usepackage{graphicx}
\usepackage{multirow}
\usepackage{graphicx}
\usepackage[export]{adjustbox}
\usepackage{mathptmx}
\usepackage{pdfpages}

% leave 1.5in margin to the left and 1in margin to the other
% sides. Don't print page number in the margin (but rather above it)
\usepackage[
backend=biber,
style=ieee,
sorting=ynt
]{biblatex}
\addbibresource{mybib.bib}

\setlength{\textheight}{8.63in}
\setlength{\textwidth}{5.9in}
\setlength{\topmargin}{-0.2in}
\setlength{\oddsidemargin}{0.3in}
\setlength{\evensidemargin}{0.3in}
\setlength{\headsep}{0.0in}
% Start to write
\begin{document}

% First things first: The Titlepage
% This is the recommended format by the library
%


% Define \brk as a command for leaving a little vertical space. Makes
% the titlepage easier to read - normally, this is NOT GOOD LATEX
% STYLE!!!
%
\newcommand{\brk}{\vspace*{0.18in}}

% No page number on the title page
\thispagestyle{empty}

% Center the whole title page
\begin{center}

\brk

% Large font and bold face for the headline. Try to keep it at one or
% two lines. Headlines over two lines will mess up the spacing, and you have to
% manually finetune it. Note that the line break in the SOURCE CODE
% does not affect the line breaking in the output file. If you want
% hardcoded line breaks, you have to mark them with a double backslash (\\)

   {\large 
  \textbf{
   Aloja: Generation of Mutated IoT-based Protocols 
  }
   }


\brk
by

\brk
% insert your name here. 
Tongwei Ren


% All this is constant:
\brk\brk
A Thesis Proposal

\brk
Submitted to the Faculty

\brk
of the 

\brk
WORCESTER POLYTECHNIC INSTITUTE
  
\brk
In partial fulfillment of the requirements for the

\brk
Degree of Master of Science

\brk
in

\brk
Computer Science

\brk


% by

% This is how LaTeX draws lines :) It's where your signature goes.
\brk\brk
\rule{3in}{1.2pt}

% Adjust this to your preferred month and year
\brk
November 2020

\end{center}

  
\vfill
\begin{center}
Advised by: Professor Lorenzo De Carli
\end{center}

\vspace{0.35in}

% Change this to your favorite CS professor.


\vspace{0.35in}

% This is also constant :)


% \begin{flushright}  
% \vspace{0.35in}
% \rule{3in}{0.8pt}
% Professor [TBD], Thesis Reader



% \end{flushright}
% Change this to your favorite CS professor.





% end of titlepage
\newpage

% This is the command for doublespacing when you use the setspace
% package
% Please do NOT use \baselinestretch, this will mess up everything,
% cause earthquakes, tornados and lots of questions for me...
% If you need a singlespaced paragraph (BAD STYLE!!!), use
% \singlespacing or \onehalfspacing and enclose it together with the
% paragraph in braces {\singlespacing This is my text... blah blah blah}
%


% From here on, we need Roman page numbers according to the library
% regulations. So let's assign those.

\pagenumbering{arabic} % or {Roman} if you like them capitalized

% The next thing is the Prefae (``Acknowledgements'').
% No standard environment for that, so we'll format it by hand.
%

% P.S. You don't have to add me to the acknowledgements for providing 
% this file :)


% And now - tataa - the text.
% This is the place to become really creative.

% From here on, we need arabic numbering again and we need to start
% from 1.



% 
% Since this is a ``report'', the topmost level of hierarchy is
% ``Chapter'', not section as you may be used to. Chapters are
% enumerated starting from 1, so Sections are 1.1, Subsections are
% 1.1.1, subsubsections don't get numbers. (You can change that, if
% you want them to be called 1.1.1.1)
%

\section{INTRODUCTION}





\setcounter{secnumdepth}{0} %% no numbering
\subsection{Related Work}

Hello \cite{10.1145/2342441.2342467}

\setcounter{secnumdepth}{1} %% Start numbering again

% \newpage
\section{METHODOLOGY}



\setcounter{secnumdepth}{2}


% The functions lists are used to check and remove spurious results in the mappings when we generate the constructor $\rightarrow$ parser mapping. The spurious results refer to functions that may appear on the mappings but not in the lists.


% Besides, assuming we already have some different mutation strategies, we also want to make the mutation dynamic, the application will 

% \newpage
\section{PRELIMINARY RESULTS}

% This part will show some results we already got or we expect to get.


% \newpage 
\section{PROPOSED WORK}



\newpage
\setcounter{secnumdepth}{0} %% no numbering
\subsection{Timeline}

\setcounter{secnumdepth}{1} %% Start numbering again

\begin{figure}[hbt!]
  \centering
  \includegraphics[scale=0.85]{Master_Thesis_Timeline.png}
%   \captionof{figure}{A sublist of the parser functions list}
\end{figure}

% \includepdf[pages={1}]{Master_Thesis_Timeline.pdf}

% Let's assume this is the end of your thesis text.

% Now come appendices, if you had any.
% Appendices are automatically numbered, just like everything else in
% LaTeX. But only after you gave this command



% Last and least (at least, that's what the library says) - the
% Bibliography.


% you can save some space by having the bibliography singlespaced, if you want
\singlespacing

%
% You should become familiar with the BibTeX program, which
% uses a *.bib-file to collect all citations that you have. It's a lot
% prettier than typing all the citations right into the document. The 
% reference to citations also works well that way, but the exact 
% explanation of that will be on the CS-GSO homepage, whenever I'll ever 
% have time for that.
%
%
% If you use BibTeX, the bibliography is very easy. You refer to
% citations in the text with \cite{tag}, where tag is the tag that you
% defined in the bib-file.
% Then, you run bibtex once in a while during compilation, and the
% rest is done in two lines:

\break
\printbibliography


% which assumes a file foo.bib in your working directory.
% The word ``Bibliography'' will appear in your document as soon as
% you used ``bibtex'' on the command line.
%
% For reference on this, refer to the CS-GSO homepage.

%============================
%That's all, folks. Have fun.
%
%                     Andreas
%============================


\end{document}









